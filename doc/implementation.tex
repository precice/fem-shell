\section{FEM Code Implementation}
contains development of the program code with focus on the assembly of the system and its solving, the process of parallelization and the coupling step with preCISE
 \subsection{Introduction to libMesh}
 was kann libMesh eigentlich alles; wo unterstützt es einen, was muss man selbst machen
 \begin{itemize}
  \item übersicht über die klassen (im sinne von: für welche probleme kann man es einsetzen)
  \item noch was allgemeines übersichtmäßiges
  \item implementierte solver z.b. für elasticity problem; hier in doxygen function description reinschauen
  \item funktionen wie assembly\_matrix muss user selbst füllen
  \item boundary\_info mit ids usw. automatisch erstellt bei mesh import
  \item boundary conditions erstellbar -> automatisch constrains system und rhs
 \end{itemize}
 \subsection{libMesh FEM}
 details about the implementation with the libmesh FEM framework:\\
 - initialization: loading of parameters, setting up libmesh (evtl. uninteressant und es kann weg, oder es muss noch mehr hier rein)\\
 - mesh loading/import: wie sieht mesh file aus bzw. welche typen werden akzeptiert, welche ids für bcs müssen verwendet werden\\
 - set up of system: erstellen des linearimplicitsystems, erstellen der variablen, der bcs, des solvers usw.\\
 - assembly of system matrix and RHS: größter teil; hier wird auf die erstellung der lokalen und globalen stiffnessmatrix eingegangen (integral mit gauss-quadratur lösen z.B. \cite{steinke2005finite} s.248), das auslesen der forces und der entsprechende eintrag in der rhs gesetzt; das mitverfolgen der bereits bearbeiteten knoten mittels unordered\_set usw.\\
 - boundary conditions: eventuell bereits unter setup; grundsätzlich auf die beiden bc-typen eingehen, wie das in libmesh gelöst wurde\\
 - solving and getting the result vector: das lösen an sich ist eine code-zeile. hier kann man aber schreiben, mit was libmesh umgehen kann an lösern, welche einstellmöglichkeiten es gibt (error-eps, \#iters). und es geht drum, wie man an die tatsächlichen werte für die displacements kommt und was daraus am ende wird\\
 - für die standalone-version noch ein absatz zur ausgabe in exodus2-file
 \subsection{Parallelization with MPI}
 additional steps to make the code ready for multi process execution with MPI\\
 - viel ist es nicht, was man tun muss, damit libmesh mit mpi läuft\\
 - grundsätzlich ist zum lösen des gleichungssystem mit mehreren prozessen petsc als externe lib notwendig\\
 - am mesh muss nichts verändert werden, da libmesh automatisch eine partitionierung des meshes vornimmt (kann aber verbessert werden)\\
 - damit rhs korrekt gesetzt wird muss über die prozessgrenzen hinweg klar sein, ob knoten bereits bearbeitet wurde oder nicht. wie das gelöst wurde kommt hier rein\newline
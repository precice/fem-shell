\section{Summary and Conclusions}
What does my code do, what problems arose, what problems persist, what does my code cannot do, where are opportunities for extensions, etc.
 \subsection{Summary}
 % was war das ziel
 % was wurde gemacht, im prinzip introduction nur in vergangenheitsform
 % keine schlussfolgerungen! das kommt er im nächsten abschnitt
 
 \subsection{Conclusion}
 % dieser teil sollte lang sein
 % es wurden verschiedene frameworks mit einander verglichen und das geignetste ausgewählt. MFEM und libMesh waren in der näheren auswahl. libMesh ist nachträglich betrachtet tatsächlich eine gute wahl gewesen. vorher wurde probiert mit MFEM zu arbeiten, aber gewisse probleme haben einen wechsel provoziert.
 % es wurden 6 finite elemente implementiert und getestet. das war nötig um einen geiegneten strukturlöser für die kopplung zu haben.
 % die einzelnen tests (nochmal) analysieren (accuracy, etc.)
 % test der convergence
 % test der parallelität
 % test der kopplung
 
 % resumé: löser geeignet für multi-physics, wenn unterteilung des meshs entsprechend hoch ist, damit accuracy stimmt. dank gut skalierender parallelisierung möglich 

 \subsection{Future Work}
  \begin{itemize}
  	\item Now: Only forces at nodes are accepted and processed. Then: Pressures linked to faces can be accepted, too. The conversion to nodal values takes place in the structure solver. (Idea from coupling data mapping)
  	\item Dynamic und Zeitabhängigkeit???????????????
  	\item Weitere Elementtypen einbauen -> Tri-6, Quad-8, usw.
  	\item Erweiterung um Frames und Beams (1D-Modelle) oder Solids (3D-Modelle)
  \end{itemize}
\newpage
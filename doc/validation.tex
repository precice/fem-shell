\section{Validation}
The code was tested with several problems to validate its correctness and state where and why there are differing results to the existing (commercial) FEM codes \textbf{TODO: besseren Text hier einfügen} %todo besseren einleitungstext formulieren, SAP 2000 benennen für später
 \subsection{Test A: Plane Displacement with Tri-3 Element}
  The three-node triangular plane element \textbf{Tri-3} is validated with a cantilever beam shown in Figure \ref{fig:testA}. The example problem was taken from \cite{kansara2004development}.
 \begin{figure}[htbp]
   	\centering
	\setlength\unitlength{1.65cm}
   	\begin{picture}(9,3)
   	\thicklines
   	\put(4.51,1.51){\vector(1,0){0.75}}
   	\put(4.52,1.5){\vector(0,1){0.75}}
   	\put(5.07,1.27){$\mathbf{x}$}
   	\put(4.55,2.13){$\mathbf{y}$}   	
   	\put(8.52,0.0){\vector(0,1){0.5}}
   	\put(8.52,1.0){\vector(0,1){0.5}}
   	\put(8.52,2.0){\vector(0,1){0.5}}   	
   	\thinlines
   	\polygon(0.5,0.5)(0.5,2.5)(8.5,2.5)(8.5,0.5)
   	\polyline(0.5,1.5)(1.5,2.5)(1.5,0.5)(0.5,1.5)(1.5,1.5)
   	\polyline(1.5,1.5)(2.5,2.5)(2.5,0.5)(1.5,1.5)(2.5,1.5)
   	\polyline(2.5,1.5)(3.5,2.5)(3.5,0.5)(2.5,1.5)(3.5,1.5)
   	\polyline(3.5,1.5)(4.5,2.5)(4.5,0.5)(3.5,1.5)(4.5,1.5)
   	\polyline(4.5,1.5)(5.5,2.5)(5.5,0.5)(4.5,1.5)(5.5,1.5)
   	\polyline(5.5,1.5)(6.5,2.5)(6.5,0.5)(5.5,1.5)(6.5,1.5)
   	\polyline(6.5,1.5)(7.5,2.5)(7.5,0.5)(6.5,1.5)(7.5,1.5)
   	\polyline(7.5,1.5)(8.5,2.5)(8.5,0.5)(7.5,1.5)(8.5,1.5)   	
   	\Line(0.5,0.5)(0.3,0.7) \Line(0.5,1)(0.3,1.2) \Line(0.5,1.5)(0.3,1.7) \Line(0.5,2)(0.3,2.2) \Line(0.5,2.5)(0.3,2.7)   	
   	\put(8.6,0.15){$6\frac{2}{3}$}
   	\put(8.6,1.15){$26\frac{2}{3}$}
   	\put(8.6,2.15){$6\frac{2}{3}$}   	
   	\put(0.56,0.55){$0$} \put(1.56,0.55){$1$} \put(2.56,0.55){$2$} \put(3.56,0.55){$3$} \put(4.56,0.55){$4$} \put(5.56,0.55){$5$} \put(6.56,0.55){$6$} \put(7.56,0.55){$7$} \put(8.56,0.55){$8$}
   	\put(0.56,1.55){$9$}  \put(1.56,1.55){$10$} \put(2.56,1.55){$11$} \put(3.56,1.55){$12$} \put(4.56,1.55){$13$} \put(5.56,1.55){$14$} \put(6.56,1.55){$15$} \put(7.56,1.55){$16$} \put(8.56,1.55){$17$}
   	\put(0.56,2.55){$18$} \put(1.56,2.55){$19$} \put(2.56,2.55){$20$} \put(3.56,2.55){$21$} \put(4.56,2.55){$22$} \put(5.56,2.55){$23$} \put(6.56,2.55){$24$} \put(7.56,2.55){$25$} \put(8.56,2.55){$26$}
   	\end{picture}
   	\caption{Cantilever beam consisting of 32 triangular elements, clamped at the left side and a total force of 40 kips applied in positive y-direction at the right side}
   	\label{fig:testA}
   \end{figure}
      
   \begin{itemize}
   \item \textbf{Mesh dimensions}\\
   Length $l = 48\ inch$\\
   Depth $h = 12\ inch$\\
   Thickness $t = 1\ inch$
   
   \item \textbf{Material properties}\\
   Young's Modulus $E = 30000 ksi$\\
   Poisson's ratio $\nu = 0.25$
   
   \item \textbf{Boundary conditions}\\
   Clamped boundary conditions at node 0, 9 and 18, i.e.\ left side of the cantilever beam.
   
   \item \textbf{Loading}\\
   A concentrated load of $40 kips$ in total. Node 8 and 26 has a load of $6 \frac{2}{3}$, node 17 has a load of $26 \frac{2}{3}$.
   \end{itemize}
   
   \paragraph{Results:} The displacements in x- and y-direction at node 22 and 26 are presented in Table \ref{tab:testA} together with the results of \textit{SAP-2000} from \cite{kansara2004development}. The displacements of the work's program deviate from the commercial software in all cases for at most $0.027\%$.
   
   \begin{table}[htbp]
   \centering
   \begin{tabular}{c|c|C{2.5cm}|C{2.5cm}|c}
   \textbf{Node} & \textbf{Displacement} & \textbf{Results from program} & \textbf{Results from SAP-2000} & \textbf{Deviation}\\\hline
   \multirow{2}{*}{22} & $u_x$ & $-0.0255988$ & $-0.025605$ & $0.024\%$\\
                       & $u_y$ & $ 0.0629549$ & $ 0.062971$ & $0.026\%$\\\hline
   \multirow{2}{*}{26} & $u_x$ & $-0.0342621$ & $-0.034271$ & $0.027\%$\\
                       & $u_y$ & $ 0.1944070$ & $ 0.194456$ & $0.025\%$\\\hline
   \end{tabular}
   \caption{Displacements and deviations for Test A}
   \label{tab:testA}
   \end{table}
   
   
 \subsection{Test B: Plane Displacement with Quad-4}
  The four-node quadrilateral plane element \textbf{Quad-4} is validated with the same cantilever beam as used in Test A. It is shown in Figure \ref{fig:testB}. The example problem was also taken from \cite{kansara2004development}.
  \begin{figure}[htbp]
    \centering
  	\setlength\unitlength{1.65cm}
   	\begin{picture}(9,3)
   	\thicklines
   	\put(4.51,1.51){\vector(1,0){0.75}}
   	\put(4.52,1.5){\vector(0,1){0.75}}
   	\put(5.07,1.27){$\mathbf{x}$}
   	\put(4.55,2.13){$\mathbf{y}$}   	
   	\put(8.52,0.0){\vector(0,1){0.5}}
   	\put(8.52,1.0){\vector(0,1){0.5}}
   	\put(8.52,2.0){\vector(0,1){0.5}}   	
   	\thinlines
   	\polygon(0.5,0.5)(0.5,2.5)(8.5,2.5)(8.5,0.5)
   	\Line(0.5,1.5)(8.5,1.5)
   	\Line(1.5,0.5)(1.5,2.5) \Line(2.5,0.5)(2.5,2.5) \Line(3.5,0.5)(3.5,2.5) \Line(4.5,0.5)(4.5,2.5) \Line(5.5,0.5)(5.5,2.5) \Line(6.5,0.5)(6.5,2.5) \Line(7.5,0.5)(7.5,2.5)
   	\Line(0.5,0.5)(0.3,0.7) \Line(0.5,1)(0.3,1.2) \Line(0.5,1.5)(0.3,1.7) \Line(0.5,2)(0.3,2.2) \Line(0.5,2.5)(0.3,2.7)   	
   	\put(8.6,0.15){$6\frac{2}{3}$}
   	\put(8.6,1.15){$26\frac{2}{3}$}
   	\put(8.6,2.15){$6\frac{2}{3}$}   	
   	\put(0.56,0.55){$0$} \put(1.56,0.55){$1$} \put(2.56,0.55){$2$} \put(3.56,0.55){$3$} \put(4.56,0.55){$4$} \put(5.56,0.55){$5$} \put(6.56,0.55){$6$} \put(7.56,0.55){$7$} \put(8.56,0.55){$8$}
   	\put(0.56,1.55){$9$}  \put(1.56,1.55){$10$} \put(2.56,1.55){$11$} \put(3.56,1.55){$12$} \put(4.56,1.55){$13$} \put(5.56,1.55){$14$} \put(6.56,1.55){$15$} \put(7.56,1.55){$16$} \put(8.56,1.55){$17$}
   	\put(0.56,2.55){$18$} \put(1.56,2.55){$19$} \put(2.56,2.55){$20$} \put(3.56,2.55){$21$} \put(4.56,2.55){$22$} \put(5.56,2.55){$23$} \put(6.56,2.55){$24$} \put(7.56,2.55){$25$} \put(8.56,2.55){$26$}
   	\end{picture}
   	\caption{Cantilever beam consisting of 16 quadrilateral elements, clamped at the left side and a total force of 40 kips applied in positive y-direction at the right side}
   	\label{fig:testB}
  \end{figure}

  \begin{itemize}
   \item \textbf{Mesh dimensions}\\
   Length $l = 48\ inch$\\
   Depth $h = 12\ inch$\\
   Thickness $t = 1\ inch$

   \item \textbf{Material properties}\\
   Young's Modulus $E = 30000 ksi$\\
   Poisson's ratio $\nu = 0.25$

   \item \textbf{Boundary conditions}\\
   Clamped boundary conditions at node 0, 9 and 18, i.e.\ left side of the cantilever beam.

   \item \textbf{Loading}\\
   A concentrated load of $40 kips$ in total. Node 8 and 26 has a load of $6 \frac{2}{3}$, node 17 has a load of $26 \frac{2}{3}$.
  \end{itemize}

  \paragraph{Results:} The displacements in x- and y-direction at node 22 and 26 are presented in Table \ref{tab:testB} together with the results of \textit{SAP-2000} from \cite{kansara2004development}. The displacements of the work's program deviate from the commercial software in all cases for at most $0.03\%$.

  \begin{table}[htbp]
   \centering
    \begin{tabular}{c|c|C{2.5cm}|C{2.5cm}|c}
    \textbf{Node} & \textbf{Displacement} & \textbf{Results from program} & \textbf{Results from SAP-2000} & \textbf{Deviation}\\\hline
    \multirow{2}{*}{22} & $u_x$ & $-0.0427728$ & $-0.042774$ & $0.028\%$\\
                        & $u_y$ & $ 0.1012620$ & $ 0.101265$ & $0.030\%$\\\hline
    \multirow{2}{*}{26} & $u_x$ & $-0.0570728$ & $-0.057074$ & $0.021\%$\\
                        & $u_y$ & $ 0.3160560$ & $ 0.316064$ & $0.025\%$\\\hline
    \end{tabular}
   \caption{Displacements and deviations for Test B}
   \label{tab:testB}
   \end{table}
     
 \subsection{Test C: Plate Displacement with Tri-3}
 %Test 4.2.1 und 4.2.2 von test_cases.pdf\\
 %Bild\\
 %Mesh: meshgen\_q\_PlateQuad nx ny 0 file q / 
 \subsection{Test D: Plate Displacement with Quad-4}
 %Selbes mesh wie Test C nur eben mit Quadelementen\newline
 %test\_a\_quadN.xda, test\_g\_quad\_N.xda - korrekt, noch nicht getestet
 \subsection{Test E: Shell Displacement}
 %Ein H-Trägerbalken. Am einen Ende fest eingespannt. Am anderen Ende wird oben eine Kraft am äußeren Knoten in den Balken hinein in flacher Ebene gegeben, gleichzeitig wird unten an der gegenüberliegenden Seite eine Kraft in entgegengesetzter Richtung gegeben\newline
 %test\_j\_tri.xda - korrekt
 %Gleich wie Test E nur eben Quadelemente\newline
 %test\_j\_quad.xda - korrekt
 \subsection{Test F: Convergence (increasing number of elements)}
 %??? theoretisch mit Test C/D bereits durchführbar mit N=2,4,8,16,32,64,128
 \subsection{Test G: MPI (increasing number of processes)}
 %??? theoretisch alle Tests, z.B. E/F mit Prozessoranzahl = 1,2,4,8,16. In dem Fall ist natürlich die Zeit interessant und ob die Ergebnisse jeweils alle gleich sind
 \subsection{Test H: Coupling with preCICE}
 %???
\newpage